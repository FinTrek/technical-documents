%%%%%%%%%%%%%%%%%%%%%%%%%%%%%%%%%%%%%%%%%
% Beamer Presentation
% LaTeX Template
% Version 1.0 (10/11/12)
%
% This template has been downloaded from:
% http://www.LaTeXTemplates.com
%
% License:
% CC BY-NC-SA 3.0 (http://creativecommons.org/licenses/by-nc-sa/3.0/)
%
%%%%%%%%%%%%%%%%%%%%%%%%%%%%%%%%%%%%%%%%%

%----------------------------------------------------------------------------------------
%	PACKAGES AND THEMES
%----------------------------------------------------------------------------------------

\documentclass[handout]{beamer}

\mode<presentation> {

% The Beamer class comes with a number of default slide themes
% which change the colors and layouts of slides. Below this is a list
% of all the themes, uncomment each in turn to see what they look like.

%\usetheme{default}
%\usetheme{AnnArbor}
%\usetheme{Antibes}
%\usetheme{Bergen}
%\usetheme{Berkeley}
%\usetheme{Berlin}
%\usetheme{Boadilla}
%\usetheme{CambridgeUS}
%\usetheme{Copenhagen}
%\usetheme{Darmstadt}
%\usetheme{Dresden}
%\usetheme{Frankfurt}
%\usetheme{Goettingen}
%\usetheme{Hannover}
%\usetheme{Ilmenau}
%\usetheme{JuanLesPins}
%\usetheme{Luebeck}
\usetheme{Madrid}
%\usetheme{Malmoe}
%\usetheme{Marburg}
%\usetheme{Montpellier}
%\usetheme{PaloAlto}
%\usetheme{Pittsburgh}
%\usetheme{Rochester}
%\usetheme{Singapore}
%\usetheme{Szeged}
%\usetheme{Warsaw}

% As well as themes, the Beamer class has a number of color themes
% for any slide theme. Uncomment each of these in turn to see how it
% changes the colors of your current slide theme.

%\usecolortheme{albatross}
%\usecolortheme{beaver}
%\usecolortheme{beetle}
%\usecolortheme{crane}
%\usecolortheme{dolphin}
%\usecolortheme{dove}
%\usecolortheme{fly}
%\usecolortheme{lily}
%\usecolortheme{orchid}
%\usecolortheme{rose}
%\usecolortheme{seagull}
%\usecolortheme{seahorse}
%\usecolortheme{whale}
%\usecolortheme{wolverine}

%\setbeamertemplate{footline} % To remove the footer line in all slides uncomment this line
%\setbeamertemplate{footline}[page number] % To replace the footer line in all slides with a simple slide count uncomment this line

%\setbeamertemplate{navigation symbols}{} % To remove the navigation symbols from the bottom of all slides uncomment this line
}

\usepackage{graphicx} % Allows including images
\usepackage{booktabs} % Allows the use of \toprule, \midrule and \bottomrule in tables
\usepackage{cool}
\usepackage{tikz}
\usepackage{amsmath}
\usepackage{pseudocode}
\DeclareMathOperator*{\argmax}{argmax}
\DeclareMathOperator*{\argmin}{argmin}
\usetikzlibrary{positioning}

%----------------------------------------------------------------------------------------
%	TITLE PAGE
%----------------------------------------------------------------------------------------

\title[Policy Gradient Algorithms]{Policy Gradient Algorithms} % The short title appears at the bottom of every slide, the full title is only on the title page

\author{Ashwin Rao} % Your name
\institute[Stanford] % Your institution as it will appear on the bottom of every slide, may be shorthand to save space
{ICME, Stanford University
 % Your institution for the title page
}

\date{} % Date, can be changed to a custom date

\begin{document}
\begin{frame}
\titlepage % Print the title page as the first slide
\end{frame}

\begin{frame}
\frametitle{Overview} % Table of contents slide, comment this block out to remove it
\tableofcontents % Throughout your presentation, if you choose to use \section{} and \subsection{} commands, these will automatically be printed on this slide as an overview of your presentation
\end{frame}

\section{Motivation and Intuition}


\begin{frame}
\frametitle{Why do we care about Policy Gradient (PG)?}
\pause
\begin{itemize}[<+->]
\item Let us review how we got here
\item We started with Markov Decision Processes and Bellman Equations
\item Next we studied several variants of DP and RL algorithms
\item We noted that the idea of {\em Generalized Policy Iteration} (GPI) is key
\item Policy Improvement step: $\pi(a|s)$ derived from $\argmax_a Q(s, a)$
\item How do we do $\argmax$ when action space is large or continuous?
\item Idea: Do Policy Improvement step with a Gradient Ascent instead
\end{itemize}
\end{frame}

\begin{frame}
\frametitle{``Policy Improvement with a Gradient Ascent??''}
\pause
\begin{itemize}[<+->]
\item We want to find the Policy that fetches the ``Best Expected Returns''
\item Gradient Ascent on ``Expected Returns'' w.r.t params of Policy func
\item So we need a func approx for (stochastic) Policy Func: $\pi(s, a; \theta)$
\item In addition to the usual func approx for Action Value Func: $Q(s, a; w)$
\item $\pi(s, a; \theta)$ func approx called {\em Actor}, $Q(s, a; w)$ func approx called {\em Critic}
\item Critic parameters $w$ are optimized w.r.t $Q(s, a; w)$ loss function $\min$
\item Actor parameters $\theta$ are optimized w.r.t Expected Returns $\max$
\item We need to formally define ``Expected Returns''
\item But we already see that this idea is appealing for continuous actions
\item GPI with Policy Improvement done as {\bf Policy Gradient (Ascent)}
\end{itemize}
\end{frame}

\begin{frame}
\frametitle{Value Function-based and Policy-based RL}
\pause
\begin{itemize}[<+->]
\item Value Function-based
\begin{itemize}
\item Learn Value Function (with a function approximation)
\item Policy is implicit - readily derived from Value Function (eg: $\epsilon$-greedy)
\end{itemize}
\item Policy-based
\begin{itemize}
\item Learn Policy (with a function approximation)
\item No requirement of a Value Function
\end{itemize}
\item Actor-Critic
\begin{itemize}
\item Learn Policy (Actor)
\item Learn Value Function (Critic)
\end{itemize}
\end{itemize}
\end{frame}

\begin{frame}
\frametitle{Advantages and Disadvantages of Policy Gradient approach}
\pause
{\bf Advantages:}
\pause
\begin{itemize}[<+->]
\item Finds the best {\em Stochastic} Policy (Optimal Deterministic Policy, produced by other RL algorithms, can be unsuitable for POMDPs)
\item Naturally {\em explores} due to Stochastic Policy representation
\item Effective in high-dimensional or continuous action spaces
\item Small changes in $\theta \Rightarrow$ small changes in $\pi$, and in state distribution
\item This avoids the convergence issues seen in $\argmax$-based algorithms 
\end{itemize}
\pause
{\bf Disadvantages:}
\pause
\begin{itemize}[<+->]
\item Typically converge to a local optimum rather than a global optimum
\item Policy Evaluation is typically inefficient and has high variance
\item Policy Improvement happens in small steps $\Rightarrow$ slow convergence
\end{itemize}
\end{frame}

\section{Definitions and Notation}

\begin{frame}
\frametitle{Notation}
\pause
\begin{itemize}[<+->]
\item Discount Factor $\gamma$
\item Assume episodic with $0 \leq \gamma \leq1$ or non-episodic with $0 \leq \gamma < 1$
\item States $s_t \in \mathcal{S}$, Actions $a_t \in \mathcal{A}$, Rewards $r_t \in \mathbb{R}$, $\forall t \in \{0, 1, 2, \ldots\}$
\item State Transition Probabilities $\mathcal{P}_{s,s'}^a = Pr(s_{t+1}=s'|s_t=s,a_t=a)$
\item Expected Rewards $\mathcal{R}_s^a = E[r_t | s_t=s,a_t=a]$
\item Initial State Probability Distribution $p_0 : \mathcal{S} \rightarrow [0,1]$
\item Policy Func Approx $\pi(s,a;\theta) = Pr(a_t=a | s_t=s,\theta), \theta \in \mathbb{R}^k$
\end{itemize}
\pause
PG coverage will be quite similar for non-discounted non-episodic, by considering average-reward objective (so we won't cover it)
\end{frame}

\begin{frame}
\frametitle{``Expected Returns" Objective}
\pause
Now we formalize the ``Expected Returns'' Objective $J(\theta)$
$$J(\theta) = \mathbb{E}_{\pi}[\sum_{t=0}^\infty \gamma^t r_t]$$
\pause
Value Function $V^{\pi}(s)$ and Action Value function $Q^{\pi}(s,a)$ defined as:
$$V^{\pi}(s) = \mathbb{E}_{\pi}[\sum_{k=t}^\infty \gamma^{k-t} r_k|s_t=s], \forall t \in \{0, 1, 2, \ldots\}$$
$$Q^{\pi}(s,a) = \mathbb{E}_{\pi}[\sum_{k=t}^\infty \gamma^{k-t} r_k|s_t=s, a_t=a], \forall t \in \{0, 1, 2, \ldots\}$$
\pause
$$\mbox{Advantage Function } A^{\pi}(s,a) = Q^{\pi}(s,a) - V^{\pi}(s)$$
\pause
Also, $p(s \rightarrow s', t, \pi)$ will be a key function for us - it denotes the probability of going from state $s$ to $s'$ in $t$ steps by following policy $\pi$
\end{frame}

\begin{frame}
\frametitle{Discounted State Visitation Measure}
\pause
$$J(\theta) = \mathbb{E}_{\pi}[\sum_{t=0}^\infty \gamma^t r_t] = \sum_{t=0}^\infty \gamma^t \mathbb{E}_{\pi}[r_t]$$
\pause
$$ = \sum_{t=0}^\infty \gamma^t \int_{\mathcal{S}} (\int_{\mathcal{S}}  p_0(s_0) \cdot p(s_0 \rightarrow s, t, \pi) \cdot ds_0) \int_{\mathcal{A}} \pi(s,a; \theta) \cdot \mathcal{R}_s^a \cdot da \cdot ds$$
\pause
$$ =  \int_{\mathcal{S}} (\int_{\mathcal{S}}  \sum_{t=0}^\infty \gamma^t \cdot p_0(s_0) \cdot p(s_0 \rightarrow s, t, \pi) \cdot ds_0) \int_{\mathcal{A}} \pi(s,a; \theta) \cdot \mathcal{R}_s^a \cdot da \cdot ds$$
\pause
\begin{definition}
$$J(\theta) =  \int_{\mathcal{S}} \rho^{\pi}(s) \int_{\mathcal{A}} \pi(s,a; \theta) \cdot \mathcal{R}_s^a \cdot da \cdot ds$$
\end{definition}
\pause
where $\rho^{\pi}(s) = \int_{\mathcal{S}}  \sum_{t=0}^\infty \gamma^t \cdot p_0(s_0) \cdot p(s_0 \rightarrow s, t, \pi) \cdot ds_0$ is the key function (for PG) that we refer to as the {\em Discounted State Visitation Measure}.
\end{frame}

\section{Policy Gradient Theorem and Proof}

\begin{frame}
\frametitle{Policy Gradient Theorem (PGT)}
\pause
\begin{theorem}
$$\nabla_{\theta} J(\theta) = \int_{\mathcal{S}} \rho^{\pi}(s) \int_{\mathcal{A}} \nabla_{\theta} \pi(s,a; \theta) \cdot Q^{\pi}(s,a) \cdot da \cdot ds$$
\end{theorem}
\pause
\begin{itemize}[<+->]
\item Note: $\rho^{\pi}(s)$ depends on $\theta$, but there's no $\nabla_{\theta} \rho^{\pi}(s)$ term in $\nabla_{\theta} J(\theta)$
\item So we can simply sample simulation paths, and at each time step, we calculate $(\nabla_{\theta} \log{\pi(s,a; \theta)}) \cdot Q^{\pi}(s,a)$ (probabilities implicit in paths)
\item Note: $\nabla_{\theta} \log{\pi(s,a; \theta)}$ is Score function (Gradient of log-likelihood)
\item We will estimate $Q^{\pi}(s,a)$ with a function approximation $Q(s,a;w)$
\item We will later show how to avoid the estimate bias of $Q(s,a;w)$
\item This numerical estimate of $\nabla_{\theta} J(\theta)$ enables {\bf Policy Gradient Ascent}
\item Let us look at the score function of some canonical $\pi(s,a; \theta)$
\end{itemize}
\end{frame}

\begin{frame}
\frametitle{Canonical $\pi(s,a;\theta)$ for finite action spaces}
\pause
\begin{itemize}[<+->]
\item For finite action spaces, we often use Softmax Policy
\item $\theta$ is an $n$-vector $(\theta_1, \ldots, \theta_n)$
\item Features vector $\phi(s,a) = (\phi_1(s,a), \ldots, \phi_n(s,a))$ for all $s \in \mathcal{S}, a \in \mathcal{A}$
\item Weight actions using linear combinations of features: $\theta^T \cdot \phi(s,a)$
\item Action probabilities proportional to exponentiated weights:
$$\pi(s,a;\theta) = \frac {e^{\theta^T \cdot \phi(s,a)}} {\sum_b e^{\theta^T \cdot \phi(s,b)}} \mbox{ for all } s \in \mathcal{S}, a \in \mathcal{A}$$
\item The score function is:
$$\nabla_{\theta} \log \pi(s,a;\theta) = \phi(s,a) - \sum_b \pi(s,b;\theta) \cdot \phi(s,b) = \phi(s,a) - \mathbb{E}_{\pi}[\phi(s,\cdot)]$$
\end{itemize}
\end{frame}

\begin{frame}
\frametitle{Canonical $\pi(s,a;\theta)$ for continuous action spaces}
\pause
\begin{itemize}[<+->]
\item For continuous action spaces, we often use Gaussian Policy
\item $\theta$ is an $n$-vector $(\theta_1, \ldots, \theta_n)$
\item State features vector $\phi(s) = (\phi_1(s), \ldots, \phi_n(s))$ for all $s \in \mathcal{S}$
\item Gaussian Mean is a linear combination of state features $\theta^T \cdot \phi(s)$
\item Variance may be fixed $\sigma^2$, or can also be parameterized
\item Policy is Gaussian, $a \sim \mathcal{N}(\theta^T \cdot \phi(s), \sigma^2) \mbox{ for all } s \in \mathcal{S}$
\item The score function is:
$$\nabla_{\theta} \log \pi(s,a;\theta) = \frac {(a - \theta^T \cdot \phi(s)) \cdot \phi(s)} {\sigma^2}$$
\end{itemize}
\end{frame}

\begin{frame}
\frametitle{Proof of Policy Gradient Theorem}
\pause
We begin the proof by noting that:
$$J(\theta) = \int_{\mathcal{S}} p_0(s_0) \cdot V^{\pi}(s_0) \cdot ds_0 = \int_{\mathcal{S}} p_0(s_0) \int_{\mathcal{A}} \pi(s_0, a_0; \theta) \cdot Q^{\pi}(s_0, a_0) \cdot da_0 \cdot ds_0$$
\pause
Calculate $\nabla_{\theta} J(\theta)$ by parts $\pi(s_0, a_0; \theta)$ and $Q^{\pi}(s_0, a_0)$
\pause
\begin{align*}
\nabla_{\theta} J(\theta) & = \int_{\mathcal{S}} p_0(s_0) \int_{\mathcal{A}} \nabla_{\theta} \pi(s_0, a_0; \theta) \cdot Q^{\pi}(s_0, a_0) \cdot da_0 \cdot ds_0 \\
& + \int_{\mathcal{S}} p_0(s_0) \int_{\mathcal{A}} \pi(s_0, a_0; \theta) \cdot \nabla_{\theta} Q^{\pi}(s_0, a_0) \cdot da_0 \cdot ds_0  \\
\end{align*}
\end{frame}

\begin{frame}
\frametitle{Proof of Policy Gradient Theorem}
\pause
Now expand $Q^{\pi}(s_0, a_0)$ as $\mathcal{R}_{s_0}^{a_0} + \int_{\mathcal{S}} \gamma \cdot \mathcal{P}_{s_0,s_1}^{a_0} \cdot V^{\pi}(s_1) \cdot ds_1$ (Bellman)
\pause
\begin{align*}
& = \int_{\mathcal{S}} p_0(s_0) \int_{\mathcal{A}} \nabla_{\theta} \pi(s_0, a_0; \theta) \cdot Q^{\pi}(s_0, a_0) \cdot da_0 \cdot ds_0 \\
& + \int_{\mathcal{S}} p_0(s_0) \int_{\mathcal{A}} \pi(s_0, a_0; \theta) \cdot \nabla_{\theta} (\mathcal{R}_{s_0}^{a_0} + \int_{\mathcal{S}} \gamma \cdot \mathcal{P}_{s_0,s_1}^{a_0} \cdot V^{\pi}(s_1) \cdot ds_1) \cdot da_0 \cdot ds_0 \\
\end{align*}
\pause
Note: $\nabla_{\theta} \mathcal{R}_{s_0}^{a_0} = 0$, so remove that term
\pause
\begin{align*}
& = \int_{\mathcal{S}} p_0(s_0) \int_{\mathcal{A}} \nabla_{\theta} \pi(s_0, a_0; \theta) \cdot Q^{\pi}(s_0, a) \cdot da_0 \cdot ds_0 \\
& + \int_{\mathcal{S}} p_0(s_0) \int_{\mathcal{A}} \pi(s_0, a_0; \theta) \cdot \nabla_{\theta} (\int_{\mathcal{S}} \gamma \cdot \mathcal{P}_{s_0,s_1}^{a_0} \cdot V^{\pi}(s_1) \cdot ds_1) \cdot da_0 \cdot ds_0 \\
\end{align*}
\end{frame}

\begin{frame}
\frametitle{Proof of Policy Gradient Theorem}
\pause
Now bring the $\nabla_{\theta}$ inside the $\int_\mathcal{S}$ to apply only on $V^{\pi}(s_1)$
\pause
\begin{align*}
& = \int_{\mathcal{S}} p_0(s_0) \int_{\mathcal{A}} \nabla_{\theta} \pi(s_0, a_0; \theta) \cdot Q^{\pi}(s_0, a) \cdot da_0 \cdot ds_0 \\
& + \int_{\mathcal{S}} p_0(s_0) \int_{\mathcal{A}} \pi(s_0, a_0; \theta) \int_{\mathcal{S}} \gamma \cdot \mathcal{P}_{s_0,s_1}^{a_0} \cdot \nabla_{\theta} V^{\pi}(s_1) \cdot ds_1 \cdot da_0 \cdot ds_0\\
\end{align*}
\pause
Now bring the outside $\int_\mathcal{S}$ and $\int_\mathcal{A}$ inside the inner $\int_{S}$
\pause
\begin{align*}
& = \int_{\mathcal{S}} p_0(s_0) \int_{\mathcal{A}} \nabla_{\theta} \pi(s_0, a_0; \theta) \cdot Q^{\pi}(s_0, a_0) \cdot da_0 \cdot ds_0 \\
& + \int_{\mathcal{S}}  (\int_{\mathcal{S}} \gamma \cdot p_0(s_0) \int_{\mathcal{A}} \pi(s_0, a_0; \theta) \cdot \mathcal{P}_{s_0,s_1}^{a_0} \cdot d{a_0} \cdot ds_0) \cdot \nabla_{\theta}V^{\pi}(s_1) \cdot ds_1\\
\end{align*}
\end{frame}

\begin{frame}
\frametitle{Policy Gradient Theorem}
\pause
Note that $\int_{\mathcal{A}} \pi(s_0, a_0; \theta) \cdot \mathcal{P}_{s_0,s_1}^{a_0} \cdot d{a_0} = p(s_0 \rightarrow s_1, 1, \pi)$
\pause
\begin{align*}
& = \int_{\mathcal{S}} p_0(s_0) \int_{\mathcal{A}} \cdot \nabla_{\theta} \pi(s_0, a_0; \theta) \cdot Q^{\pi}(s_0, a_0) \cdot da_0 \cdot ds_0 \\
& + \int_{\mathcal{S}} (\int_{\mathcal{S}} \gamma \cdot p_0(s_0) \cdot p(s_0 \rightarrow s_1, 1, \pi) \cdot ds_0) \cdot  \nabla_{\theta} V^{\pi}(s_1) \cdot ds_1\\
\end{align*}
\pause
Now expand $V^{\pi}(s_1)$ to $\int_{\mathcal{A}} \pi(s_1, a_1; \theta) \cdot Q^{\pi}(s_1,a_1) \cdot da_1$
\pause
\begin{align*}
& = \int_{\mathcal{S}} p_0(s_0) \int_{\mathcal{A}} \cdot \nabla_{\theta} \pi(s_0, a_0; \theta) \cdot Q^{\pi}(s_0, a_0) \cdot da \cdot ds_0 \\
& + \int_{\mathcal{S}} (\int_{\mathcal{S}} \gamma \cdot p_0(s_0) p(s_0 \rightarrow s_1, 1, \pi) ds_0) \cdot \nabla_{\theta} (\int_{\mathcal{A}} \pi(s_1, a_1; \theta) Q^{\pi}(s_1,a_1) da_1) ds_1\\
\end{align*}
\end{frame}

\begin{frame}
\frametitle{Proof of Policy Gradient Theorem}
\pause
We are now back to when we started calculating gradient of $\int_{\mathcal{A}} \pi \cdot Q^{\pi} \cdot da$.\\
\pause
Follow the same process of splitting $\pi \cdot Q^{\pi}$, then Bellman-expanding $Q^{\pi}$ (to calculate its gradient), and iterate.
\pause
\begin{align*}
& = \int_{\mathcal{S}} p_0(s_0) \int_{\mathcal{A}} \cdot \nabla_{\theta} \pi(s_0, a_0; \theta) \cdot Q^{\pi}(s_0, a_0) \cdot da_0 \cdot ds_0 \\
& + \int_{\mathcal{S}} \int_{\mathcal{S}} \gamma p_0(s_0) p(s_0 \rightarrow s_1, 1, \pi) ds_0 (\int_{\mathcal{A}} \nabla_{\theta} \pi(s_1, a_1; \theta) Q^{\pi}(s_1,a_1) da_1 + \ldots)ds_1\\
\end{align*}
\pause
This iterative process leads us to:
$$= \sum_{t=0}^\infty \int_{\mathcal{S}} \int_{\mathcal{S}} \gamma^t \cdot p_0(s_0) \cdot p(s_0 \rightarrow s_t, t, \pi) \cdot ds_0 \int_{\mathcal{A}}\nabla_{\theta} \pi(s_t, a_t; \theta) \cdot Q^{\pi}(s_t,a_t) \cdot da_t \cdot ds_t$$
\end{frame}

\begin{frame}
\frametitle{Proof of Policy Gradient Theorem}
\pause
Bring $\sum_{t=0}^{\infty}$ inside the two $\int_{\mathcal{S}}$, and note that $\int_{\mathcal{A}} \nabla_{\theta} \pi(s_t, a_t; \theta) \cdot Q^{\pi}(s_t,a_t) \cdot da_t$ is independent of $t$.
\pause
$$= \int_{\mathcal{S}} \int_{\mathcal{S}} \sum_{t=0}^{\infty} \gamma^t \cdot p_0(s_0) \cdot p(s_0 \rightarrow s, t, \pi) \cdot ds_0 \int_{\mathcal{A}} \nabla_{\theta} \pi(s, a; \theta) \cdot Q^{\pi}(s,a) \cdot da \cdot ds$$
\pause
Reminder that $\int_{\mathcal{S}} \sum_{t=0}^{\infty} \gamma^t \cdot p_0(s_0) \cdot p(s_0 \rightarrow s, t, \pi) \cdot ds_0 \overset{\mathrm{def}}{=} \rho^{\pi}(s)$. So,
\pause
$$ \nabla_{\theta} J(\theta) = \int_{\mathcal{S}} \rho^{\pi}(s) \int_{\mathcal{A}} \nabla_{\theta} \pi(s, a; \theta) \cdot Q^{\pi}(s,a) \cdot da \cdot ds $$
$$\mathbb{Q.E.D.}$$
\end{frame}

\section{Policy Gradient Algorithms}

\begin{frame}
\frametitle{Monte-Carlo Policy Gradient (REINFORCE Algorithm)}
\pause
\begin{itemize}[<+->]
\item Update $\theta$ by stochastic gradient ascent using PGT
\item Using $G_t = \sum_{k=t}^T \gamma^{k-t} \cdot r_k$ as an unbiased sample of $Q^{\pi}(s_t,a_t)$
$$\Delta \theta_t = \alpha \cdot \gamma^t \cdot \nabla_{\theta} \log \pi(s_t, a_t; \theta) \cdot G_t$$
\end{itemize}
\pause
\begin{pseudocode}{REINFORCE}{\cdot}
\mbox{Initialize } \theta \mbox{ arbitrarily}\\
\FOR \mbox{each episode } \{s_0, a_0, r_0, \ldots, s_T, a_T, r_T\} \sim \pi(\cdot, \cdot; \theta) \DO
\BEGIN
\FOR t \GETS 0 \TO T \DO
\BEGIN
G \GETS \sum_{k=t}^{T} \gamma^{k-t} \cdot r_k\\
\theta \GETS \theta + \alpha \cdot \gamma^t \cdot \nabla_{\theta} \log \pi(s_t, a_t; \theta) \cdot G\\
\END\\
\END\\
\RETURN \theta
\end{pseudocode}
\end{frame}



\begin{frame}
\frametitle{Reducing Variance using a Critic}
\pause
\begin{itemize}[<+->]
\item Monte Carlo Policy Gradient has high variance
\item We use a Critic $Q(s,a; w)$ to estimate $Q^{\pi}(s,a)$
\item Actor-Critic algorithms maintain two sets of parameters:
\begin{itemize}
\item Critic updates parameters $w$ to approximate $Q$-function for policy $\pi$
\item Critic could use any of the algorithms we learnt earlier:
\begin{itemize}
\item Monte Carlo policy evaluation
\item Temporal-Difference Learning
\item $TD(\lambda)$ based on Eligibility Traces
\item Could even use LSTD (if critic function approximation is linear)
\end{itemize}
\item Actor updates policy parameters $\theta$ in direction suggested by Critic
\item This is Approximate Policy Gradient due to {\em Bias} of Critic
$$ \nabla_{\theta} J(\theta) \approx \int_{\mathcal{S}} \rho^{\pi}(s) \int_{\mathcal{A}} \nabla_{\theta} \pi(s, a; \theta) \cdot Q(s,a; w) \cdot da \cdot ds $$
\end{itemize}
\end{itemize}
\end{frame}

\begin{frame}
\frametitle{So what does the algorithm look like?}
\pause
\begin{itemize}[<+->]
\item Generate a sufficient set of simulation paths $s_0,a_0,r_0,s_1,a_1,r_1,\ldots$
\item $s_0$ is sampled from the distribution $p_0(\cdot)$
\item $a_t$ is sampled from $\pi(s_t,\cdot; \theta)$
\item $s_{t+1}$ sampled from transition probs and $r_{t+1}$ from reward func
\item At each time step $t$, update $w$ proportional to gradient of appropriate (MC or TD-based) loss function of $Q(s,a;w)$
\item Sum $\gamma^t \cdot \nabla_{\theta} \log \pi(s_t,a_t; \theta) \cdot Q(s_t,a_t; w)$ over $t$ and over paths
\item Update $\theta$ using this (biased) estimate of $\nabla_{\theta} J(\theta)$
\item Iterate with a new set of simulation paths ...
\end{itemize}
\end{frame}

\begin{frame}
\frametitle{Reducing Variance with a Baseline}
\pause
\begin{itemize}[<+->]
\item We can reduce variance by subtracting a baseline function $B(s)$ from $Q(s,a;w)$ in the Policy Gradient estimate
\item This means at each time step, we replace $\gamma^t \cdot \nabla_{\theta} \log \pi(s_t,a_t; \theta) \cdot Q(s_t,a_t; w)$ with $\gamma^t \cdot \nabla_{\theta} \log \pi(s_t,a_t; \theta) \cdot (Q(s_t,a_t; w) - B(s))$
\item Note that Baseline function $B(s)$ is only a function of $s$ (and not $a$)
\item This ensures that subtracting Baseline $B(s)$ does not add bias
\begin{align*}
& \int_{\mathcal{S}} \rho^{\pi}(s) \int_{\mathcal{A}} \nabla_{\theta} \pi(s,a; \theta) \cdot B(s)\cdot da \cdot ds\\
 = & \int_{\mathcal{S}} \rho^{\pi}(s) \cdot B(s) \cdot \nabla_{\theta} (\int_{\mathcal{A}} \pi(s,a; \theta) \cdot da) \cdot ds = 0\\
\end{align*}
\end{itemize}
\end{frame}

\begin{frame}
\frametitle{Using State Value function as Baseline}
\pause
\begin{itemize}[<+->]
\item A good baseline $B(s)$ is state value function $V(s;v)$
\item Rewrite Policy Gradient algorithm using advantage function estimate
$$A(s,a;w,v) = Q(s,a;w) - V(s; v)$$
\item Now the estimate of $\nabla_{\theta} J(\theta)$ is given by:
$$\int_{\mathcal{S}} \rho^{\pi}(s) \int_{\mathcal{A}} \nabla_{\theta} \pi(s, a; \theta) \cdot A(s,a; w, v) \cdot da \cdot ds $$
\item At each time step, we update both sets of parameters $w$ and $v$
\end{itemize}
\end{frame}

\begin{frame}
\frametitle{TD Error as estimate of Advantage Function}
\pause
\begin{itemize}[<+->]
\item Consider TD error $\delta^{\pi}$ for the {\em true} Value Function $V^{\pi}(s)$
$$\delta^{\pi} = r + \gamma V^{\pi}(s') - V^{\pi}(s)$$
\item $\delta^{\pi}$ is an unbiased estimate of Advantage function $A^{\pi}(s,a)$
$$\mathbb{E}_{\pi}[\delta^{\pi} | s,a] = \mathbb{E}_{\pi}[r + \gamma V^{\pi}(s') | s, a] - V^{\pi}(s) = Q^{\pi}(s,a) - V^{\pi}(s) = A^{\pi}(s,a)$$
\item So we can write Policy Gradient in terms of $\mathbb{E}_{\pi}[\delta^{\pi} | s,a]$
$$\nabla_{\theta} J(\theta) = \int_{\mathcal{S}} \rho^{\pi}(s) \int_{\mathcal{A}} \nabla_{\theta} \pi(s, a; \theta) \cdot \mathbb{E}_{\pi}[\delta^{\pi} | s,a] \cdot da \cdot ds$$
\item In practice, we can use func approx for TD error (and sample):
$$\delta(s,r,s';v) = r + \gamma V(s';v) - V(s;v)$$
\item This approach requires only one set of critic parameters $v$
\end{itemize}
\end{frame}

\begin{frame}
\frametitle{TD Error can be used by both Actor and Critic}
\pause
\begin{pseudocode}{ACTOR-CRITIC-TD-ERROR}{\cdot}
\mbox{Initialize Policy params } \theta \in \mathbb{R}^m \mbox{ and State VF params } v \in \mathbb{R}^n \mbox{ arbitrarily}\\
\FOR \mbox{each episode }  \DO
\BEGIN
\mbox{Initialize } s \mbox{ (first state of episode)}\\
P \GETS 1\\
\WHILE s \mbox{ is not terminal} \DO
\BEGIN
a \sim \pi(s, \cdot; \theta)\\
\mbox{Take action } a, \mbox{ observe } r, s'\\
\delta \GETS r + \gamma V(s'; v) - V(s; v)\\
v \GETS v + \alpha_v \cdot \delta \cdot \nabla_v V(s; v)\\
\theta \GETS \theta + \alpha_{\theta} \cdot P \cdot \delta \cdot \nabla_{\theta} \log \pi(s, a; \theta)\\
P \GETS \gamma P\\
s \GETS s' \\
\END\\
\END\\
\end{pseudocode}

\end{frame}

\begin{frame}
\frametitle{Using Eligibility Traces for both Actor and Critic}
\pause
\begin{pseudocode}{ACTOR-CRITIC-ELIGIBILITY-TRACES}{\cdot}
\mbox{Initialize Policy params } \theta \in \mathbb{R}^m \mbox{ and State VF params } v \in \mathbb{R}^n \mbox{ arbitrarily}\\
\FOR \mbox{each episode }  \DO
\BEGIN
\mbox{Initialize } s \mbox{ (first state of episode)}\\
z_{\theta}, z_v \GETS 0 \mbox{ (} m \mbox{ and } n \mbox{ components eligibility trace vectors})\\
P \GETS 1\\
\WHILE s \mbox{ is not terminal} \DO
\BEGIN
a \sim \pi(s, \cdot; \theta)\\
\mbox{Take action } a, \mbox{ observe } r, s'\\
\delta \GETS r + \gamma V(s'; v) - V(s; v)\\
z_v \GETS \gamma \cdot \lambda_v \cdot z_v + \nabla_v V(s;v)\\
z_{\theta} \GETS \gamma \cdot \lambda_{\theta} \cdot z_{\theta} + P \cdot \nabla_{\theta} \log \pi(s,a;\theta)\\
v \GETS v + \alpha_v \cdot \delta \cdot z_v\\
\theta \GETS \theta + \alpha_{\theta} \cdot \delta \cdot z_{\theta}\\
P \GETS \gamma P, s \GETS s'\\
\END\\
\END\\
\end{pseudocode}

\end{frame}



\section{Compatible Function Approximation Theorem and Proof}

\begin{frame}
\frametitle{Overcoming Bias}
\pause
\begin{itemize}[<+->]
\item We've learnt a few ways of how to reduce variance
\item But we haven't discussed how to overcome bias
\item All of the following substitutes for $Q^{\pi}(s,a)$ in PG have bias:
\begin{itemize}
\item $Q(s,a;w)$
\item $A(s,a;w, v)$
\item $\delta(s,s',r;v)$
\end{itemize}
\item Turns out there is indeed a way to overcome bias
\item It is called the  {\em Compatible Function Approximation Theorem}
\end{itemize}	
\end{frame}

\begin{frame}
\frametitle{Compatible Function Approximation Theorem}
\pause
\begin{theorem}
If the following two conditions are satisfied:
\pause
\begin{enumerate}
\item Critic gradient is {\em compatible} with the Actor score function
$$\nabla_w Q(s,a;w) = \nabla_{\theta} \log \pi(s,a; \theta)$$ 
\pause
\item Critic parameters $w$ minimize the following mean-squared error:
$$\epsilon = \int_{\mathcal{S}} \rho^{\pi}(s) \int_{\mathcal{A}} \pi(s,a; \theta) (Q^{\pi}(s,a) - Q(s,a;w))^2 \cdot da \cdot ds$$
\end{enumerate}
\pause
Then the Policy Gradient using critic $Q(s,a;w)$ is exact:
$$\nabla_{\theta} J(\theta) = \int_{\mathcal{S}} \rho^{\pi}(s) \int_{\mathcal{A}} \nabla_{\theta} \pi(s, a; \theta) \cdot Q(s,a; w) \cdot da \cdot ds$$
\end{theorem}
\end{frame}

\begin{frame}
\frametitle{Proof of Compatible Function Approximation Theorem}
\pause
For $w$ that minimizes
$$\epsilon = \int_{\mathcal{S}} \rho^{\pi}(s) \int_{\mathcal{A}} \pi(s,a; \theta) \cdot (Q^{\pi}(s,a) - Q(s,a;w))^2 \cdot da \cdot ds,$$
\pause
$$\int_{\mathcal{S}} \rho^{\pi}(s) \int_{\mathcal{A}} \pi(s,a; \theta) \cdot (Q^{\pi}(s,a) - Q(s,a;w)) \cdot \nabla_w Q(s,a;w) \cdot da \cdot ds = 0$$
\pause
But since $\nabla_w Q(s,a;w) = \nabla_{\theta} \log \pi(s,a; \theta)$, we have:
\pause
$$\int_{\mathcal{S}} \rho^{\pi}(s) \int_{\mathcal{A}} \pi(s,a; \theta) \cdot (Q^{\pi}(s,a) - Q(s,a;w)) \cdot \nabla_{\theta} \log \pi(s,a; \theta) \cdot da \cdot ds = 0$$
\pause
\begin{align*}
\mbox{Therefore, } & \int_{\mathcal{S}} \rho^{\pi}(s) \int_{\mathcal{A}} \pi(s,a; \theta) \cdot Q^{\pi}(s,a) \cdot \nabla_{\theta} \log \pi(s,a; \theta) \cdot da \cdot ds \\
= & \int_{\mathcal{S}} \rho^{\pi}(s) \int_{\mathcal{A}} \pi(s,a; \theta) \cdot Q(s,a; w) \cdot \nabla_{\theta} \log \pi(s,a; \theta) \cdot da \cdot ds\\
\end{align*}
\end{frame}

\begin{frame}
\frametitle{Proof of Compatible Function Approximation Theorem}
\pause
$$\mbox{But } \nabla_{\theta} J(\theta) = \int_{\mathcal{S}} \rho^{\pi}(s) \int_{\mathcal{A}} \pi(s,a; \theta) \cdot Q^{\pi}(s,a) \cdot \nabla_{\theta} \log \pi(s,a; \theta) \cdot da \cdot ds$$
\pause
\begin{align*}
\mbox{So, } \nabla_{\theta} J(\theta) & = \int_{\mathcal{S}} \rho^{\pi}(s) \int_{\mathcal{A}} \pi(s,a; \theta) \cdot Q(s,a; w) \cdot \nabla_{\theta} \log \pi(s,a; \theta) \cdot da \cdot ds\\
& = \int_{\mathcal{S}} \rho^{\pi}(s) \int_{\mathcal{A}} \nabla_{\theta} \pi(s,a; \theta) \cdot Q(s,a; w) \cdot da \cdot ds\\
\end{align*}
$$\mathbb{Q.E.D.}$$
\pause
{\bf This means with conditions (1) and (2) of Compatible Function Approximation Theorem, we can use the critic func approx $Q(s,a;w)$ and still have the exact Policy Gradient.}
\end{frame}



\begin{frame}
\frametitle{How to enable Compatible Function Approximation}
\pause
A simple way to enable Compatible Function Approximation
\pause
$\pderiv{Q(s,a;w)}{w_i} = \pderiv{\log \pi(s,a; \theta)}{\theta_i}, \forall i$ is to set $Q(s,a;w)$ to be linear in its features.
\pause
$$Q(s,a;w) = \sum_{i=1}^n \phi_i(s,a) \cdot w_i = \sum_{i=1}^n \pderiv{\log \pi(s,a;\theta)}{\theta_i} \cdot w_i$$
\pause
We note below that a compatible $Q(s,a;w)$ serves as an approximation of the advantage function.
\pause
$$\int_{\mathcal{A}} \pi(s,a;\theta) \cdot Q(s,a;w) \cdot da = \int_{\mathcal{A}} \pi(s,a;\theta) \cdot (\sum_{i=1}^n \pderiv{\log \pi(s,a;\theta)}{\theta_i} \cdot w_i) \cdot da$$
\pause
$$=\int_{\mathcal{A}} \cdot (\sum_{i=1}^n \pderiv{\pi(s,a;\theta)}{\theta_i} \cdot w_i) \cdot da = \sum_{i=1}^n (\int_{\mathcal{A}} \pderiv{\pi(s,a;\theta)}{\theta_i} \cdot da)\cdot w_i$$
\pause
$$=\sum_{i=1}^n \frac {\partial} {\partial \theta_i} (\int_{\mathcal{A}} \pi(s,a;\theta) \cdot da)\cdot w_i = \sum_{i=1}^n \pderiv{1}{\theta_i} \cdot w_i = 0$$
\end{frame}

\section{Natural Policy Gradient}

\begin{frame}
\frametitle{Fisher Information Matrix}
\pause
Denoting $[\pderiv{\log \pi(s,a;\theta)}{\theta_i}], i = 1, \ldots, n$ as the score column vector $SC(s,a;\theta)$ and assuming compatible linear-approx critic:
\pause
\begin{align*}
\nabla_{\theta} J(\theta) & = \int_{\mathcal{S}} \rho^{\pi}(s) \int_{\mathcal{A}} \pi(s,a;\theta) \cdot (SC(s, a; \theta) \cdot SC(s,a;\theta)^T \cdot w) \cdot da \cdot ds\\
& = E_{s \sim \rho^{\pi}, a \sim \pi}[SC(s,a;\theta) \cdot SC(s,a;\theta)^T] \cdot w\\
& = FIM_{\rho^{\pi}, \pi}(\theta) \cdot w\\
\end{align*}
\pause
where $FIM_{\rho_{\pi}, \pi}(\theta)$ is the Fisher Information Matrix w.r.t. $s \sim \rho^{\pi}, a \sim \pi$.
\end{frame}

\begin{frame}
\frametitle{Natural Policy Gradient}
\pause
\begin{itemize}[<+->]
\item Recall the idea of Natural Gradient from Numerical Optimization
\item Natural gradient $\nabla_{\theta}^{nat} J(\theta)$ is the direction of optimal $\theta$ movement
\item In terms of the KL-divergence metric (versus plain Euclidean norm)
\item Natural gradient yields better convergence (we won't cover proof)
\end{itemize}
\pause
$$\mbox{Formally defined as: } \nabla_{\theta} J(\theta) = FIM_{\rho_{\pi}, \pi}(\theta) \cdot \nabla_{\theta}^{nat} J(\theta) $$
\pause
$$\mbox{Therefore, } \nabla_{\theta}^{nat} J(\theta) = w$$
\pause
{\bf This compact result is great for our algorithm:}
\pause
\begin{itemize}[<+->]
\item Update Critic params $w$ with the critic loss gradient (at step $t$) as:
$$\gamma^t \cdot (r_t + \gamma \cdot SC(s_{t+1}, a_{t+1},\theta) \cdot w - SC(s_t,a_t,\theta) \cdot w) \cdot SC(s_t,a_t,\theta)$$
\item Update Actor params $\theta$ in the direction equal to value of $w$
\end{itemize}
\end{frame}

\end{document}