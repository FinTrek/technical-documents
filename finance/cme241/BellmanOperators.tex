%%%%%%%%%%%%%%%%%%%%%%%%%%%%%%%%%%%%%%%%%
% Beamer Presentation
% LaTeX Template
% Version 1.0 (10/11/12)
%
% This template has been downloaded from:
% http://www.LaTeXTemplates.com
%
% License:
% CC BY-NC-SA 3.0 (http://creativecommons.org/licenses/by-nc-sa/3.0/)
%
%%%%%%%%%%%%%%%%%%%%%%%%%%%%%%%%%%%%%%%%%

%----------------------------------------------------------------------------------------
%	PACKAGES AND THEMES
%----------------------------------------------------------------------------------------

\documentclass{beamer}

\mode<presentation> {

% The Beamer class comes with a number of default slide themes
% which change the colors and layouts of slides. Below this is a list
% of all the themes, uncomment each in turn to see what they look like.

%\usetheme{default}
%\usetheme{AnnArbor}
%\usetheme{Antibes}
%\usetheme{Bergen}
%\usetheme{Berkeley}
%\usetheme{Berlin}
%\usetheme{Boadilla}
%\usetheme{CambridgeUS}
%\usetheme{Copenhagen}
%\usetheme{Darmstadt}
%\usetheme{Dresden}
%\usetheme{Frankfurt}
%\usetheme{Goettingen}
%\usetheme{Hannover}
%\usetheme{Ilmenau}
%\usetheme{JuanLesPins}
%\usetheme{Luebeck}
\usetheme{Madrid}
%\usetheme{Malmoe}
%\usetheme{Marburg}
%\usetheme{Montpellier}
%\usetheme{PaloAlto}
%\usetheme{Pittsburgh}
%\usetheme{Rochester}
%\usetheme{Singapore}
%\usetheme{Szeged}
%\usetheme{Warsaw}

% As well as themes, the Beamer class has a number of color themes
% for any slide theme. Uncomment each of these in turn to see how it
% changes the colors of your current slide theme.

%\usecolortheme{albatross}
%\usecolortheme{beaver}
%\usecolortheme{beetle}
%\usecolortheme{crane}
%\usecolortheme{dolphin}
%\usecolortheme{dove}
%\usecolortheme{fly}
%\usecolortheme{lily}
%\usecolortheme{orchid}
%\usecolortheme{rose}
%\usecolortheme{seagull}
%\usecolortheme{seahorse}
%\usecolortheme{whale}
%\usecolortheme{wolverine}

%\setbeamertemplate{footline} % To remove the footer line in all slides uncomment this line
%\setbeamertemplate{footline}[page number] % To replace the footer line in all slides with a simple slide count uncomment this line

%\setbeamertemplate{navigation symbols}{} % To remove the navigation symbols from the bottom of all slides uncomment this line
}

\usepackage{graphicx} % Allows including images
\usepackage{booktabs} % Allows the use of \toprule, \midrule and \bottomrule in tables
\usepackage{cool}
\usepackage{amsmath}

\newcommand{\vpi}{{\bf v_{\pi}}}
\newcommand{\vstar}{{\bf v_*}}
\newcommand{\bbp}{{\bf B}_{\pi}}
\newcommand{\bbs}{{\bf B}_*}
\newcommand{\bv}{{\bf v}}
\newcommand{\pis}{\pi_*}
\DeclareMathOperator*{\argmin}{arg\,min}
\DeclareMathOperator*{\argmax}{arg\,max}

\newcounter{sauvegardeenumi}
\newcommand{\asuivre}{\setcounter{sauvegardeenumi}{\theenumi}}
\newcommand{\suite}{\setcounter{enumi}{\thesauvegardeenumi}}

%----------------------------------------------------------------------------------------
%	TITLE PAGE
%----------------------------------------------------------------------------------------

\title[Bellman Operators]{Understanding (Exact) Dynamic Programming through Bellman Operators} % The short title appears at the bottom of every slide, the full title is only on the title page

\author{Ashwin Rao} % Your name
\institute[Stanford] % Your institution as it will appear on the bottom of every slide, may be shorthand to save space
{
ICME, Stanford University
}

\date{\today} % Date, can be changed to a custom date

\begin{document}
\begin{frame}
\titlepage % Print the title page as the first slide
\end{frame}

\begin{frame}
\frametitle{Overview} % Table of contents slide, comment this block out to remove it
\tableofcontents % Throughout your presentation, if you choose to use \section{} and \subsection{} commands, these will automatically be printed on this slide as an overview of your presentation
\end{frame}

\section{Vector Space of Value Functions}

\begin{frame}
\frametitle{Vector Space of Value Functions}
\begin{itemize}
\item Assume State pace $\mathcal{S}$ consists of $n$ states: $\{s_1, s_2, \ldots, s_n\}$
\item Assume Action space $\mathcal{A}$ consists of $m$ actions $\{a_1, a_2, \ldots, a_m\}$ 
\item This exposition extends easily to continuous state/action spaces too
\item We denote a stochastic policy as $\pi(a|s)$ (probability of ``$a$ given $s$'')
\item Abusing notation, deterministic policy denoted as $\pi(s) = a$
\item Consider a $n$-dim vector space, each dim corresponding to a state in $\mathcal{S}$
\item A vector in this space is a specific Value Function (VF) $\bv$: $\mathcal{S} \rightarrow \mathbb{R}$
\item With coordinates $[\bv(s_1), \bv(s_2), \ldots, \bv(s_n)]$
\item Value Function (VF) for a policy $\pi$ is denoted as $\vpi: \mathcal{S} \rightarrow \mathbb{R}$
\item Optimal VF denoted as $\vstar: \mathcal{S} \rightarrow \mathbb{R}$ such that for any $s \in \mathcal{S}$,
$$\vstar(s) = \max_{\pi} \vpi(s)$$
\end{itemize}
\end{frame}

\begin{frame}
\frametitle{Some more notation}
\begin{itemize}
\item Denote $\mathcal{R}_s^a$ as the Expected Reward upon action $a$ in state $s$
\item Denote $\mathcal{P}_{s,s'}^a$ as the probability of transition $s \rightarrow s'$ upon action $a$
\item Define
$${\bf R}_{\pi}(s) = \sum_{a \in \mathcal{A}} \pi(a|s) \cdot \mathcal{R}_s^a$$
$${\bf P}_{\pi}(s,s') = \sum_{a \in \mathcal{A}} \pi(a|s) \cdot \mathcal{P}_{s,s'}^a$$
\item Denote ${\bf R}_{\pi}$ as the vector $[{\bf R}_{\pi}(s_1), {\bf R}_{\pi}(s_2), \ldots, {\bf R}_{\pi}(s_n)]$
\item Denote ${\bf P}_{\pi}$ as the matrix $[{\bf P}_{\pi}(s_i, s_{i'})], 1 \leq i, i' \leq n$ 
\item Denote $\gamma$ as the MDP discount factor
\end{itemize}
\end{frame}

\section{Bellman Operators}

\begin{frame}
\frametitle{Bellman Operators $\bbp$ and $\bbs$}
\begin{itemize}
\item We define operators that transform a VF vector to another VF vector
\item {\em Bellman Policy Operator} $\bbp$ (for policy $\pi$) operating on VF vector $\bv$:
$$\bbp \bv = {\bf R}_{\pi} + \gamma {\bf P}_{\pi} \cdot \bv$$
\item $\bbp$ is a linear operator with fixed point $\vpi$, meaning $\bbp \vpi = \vpi$
\item {\em Bellman Optimality Operator} $\bbs$ operating on VF vector $\bv$:
$$(\bbs \bv)(s) = \max_a \{ \mathcal{R}_s^a + \gamma \sum_{s' \in \mathcal{S}} \mathcal{P}_{s,s'}^a \cdot \bv(s') \}$$
\item $\bbs$ is a non-linear operator with fixed point  $\vstar$, meaning $\bbs \vstar = \vstar$
\item Define a function $G$ mapping a VF $\bv$ to a deterministic ``greedy'' policy $G(\bv)$ as follows:
$$G(\bv)(s) = \argmax_a \{ \mathcal{R}_s^a + \gamma \sum_{s' \in \mathcal{S}} \mathcal{P}_{s,s'}^a \cdot \bv(s') \}$$
\item ${\bf B}_{G(\bv)} \bv= \bbs \bv$ for any VF $\bv$ (Policy $G(\bv)$ achieves the $\max$ in $\bbs$)
\end{itemize}
\end{frame}

\section{Contraction and Monotonicity}

\begin{frame}
\frametitle{Contraction and Monotonicity of Operators}
\begin{itemize}
\item Both $\bbp$ and $\bbs$ are $\gamma$-contraction operators in $L^{\infty}$ norm, meaning:
\item For any two VFs ${\bf v_1}$ and ${\bf v_2}$,
$$\Vert \bbp {\bf v_1} - \bbp {\bf v_2} \Vert_{\infty} \leq \gamma \Vert {\bf v_1} - {\bf v_2} \Vert_{\infty}$$
$$\Vert \bbs {\bf v_1} - \bbs {\bf v_2} \Vert_{\infty} \leq \gamma \Vert {\bf v_1} - {\bf v_2} \Vert_{\infty}$$
\item So we can invoke Contraction Mapping Theorem to claim fixed point
\item We use the notation ${\bf v_1} \leq {\bf v_2}$ for any two VFs ${\bf v_1}, {\bf v_2}$ to mean:
$${\bf v_1}(s) \leq {\bf v_2}(s) \mbox{ for all } s \in \mathcal{S}$$
\item Also, both $\bbp$ and $\bbs$ are monotonic, meaning:
\item For any two VFs ${\bf v_1}$ and ${\bf v_2}$,
$${\bf v_1} \leq {\bf v_2} \Rightarrow \bbp {\bf v_1} \leq \bbp {\bf v_2}$$
$${\bf v_1} \leq {\bf v_2} \Rightarrow \bbs {\bf v_1} \leq \bbs {\bf v_2}$$  
\end{itemize}
\end{frame}

\section{Policy Evaluation}

\begin{frame}
\frametitle{Policy Evaluation}
\begin{itemize}
\item $\bbp$ satisfies the conditions of Contraction Mapping Theorem
\item $\bbp$ has a unique fixed point $\vpi$, meaning $\bbp \vpi = \vpi$
\item This is a succinct representation of Bellman Expectation Equation
\item Starting with any VF $\bv$ and repeatedly applying $\bbp$, we will reach $\vpi$
$$\lim_{N\rightarrow \infty} \bbp^N \bv = \vpi \mbox{ for any VF } \bv$$
\item This is a succinct representation of the Policy Evaluation Algorithm
\end{itemize}
\end{frame}

\section{Policy Iteration}

\begin{frame}
\frametitle{Policy Improvement}
\begin{itemize}
\item Let $\pi_k$ and ${\bf v_{\pi_k}}$ denote the Policy and the VF for the Policy in iteration $k$ of Policy Iteration
\item Policy Improvement Step is: $\pi_{k+1} = G({\bf v_{\pi_k}})$, i.e. deterministic greedy
\item Earlier we argued that $\bbs \bv = {\bf B}_{G(\bv)} \bv$ for any VF $\bv$. Therefore,
\begin{equation}
\bbs {\bf v_{\pi_k}} = {\bf B}_{G({\bf v_{\pi_k}})} {\bf v_{\pi_k}} = {\bf B}_{\pi_{k+1}} {\bf v_{\pi_k}}\label{eq:1}
\end{equation}
\item We also know from operator definitions that $\bbs \bv \geq \bbp \bv$ for all $\pi, \bv$
\begin{equation}
\bbs {\bf v_{\pi_k}} \geq  {\bf B}_{\pi_k} {\bf v_{\pi_k}} = {\bf v_{\pi_k}}\label{eq:2}
\end{equation}
\item Combining \eqref{eq:1} and \eqref{eq:2}, we get:
$${\bf B}_{\pi_{k+1}} {\bf v_{\pi_k}} \geq {\bf v_{\pi_k}}$$
\item Monotonicity of ${\bf B}_{\pi_{k+1}}$ implies 
$${\bf B}_{\pi_{k+1}}^N {\bf v_{\pi_k}} \geq \ldots {\bf B}_{\pi_{k+1}}^2 {\bf v_{\pi_k}} \geq {\bf B}_{\pi_{k+1}} {\bf v_{\pi_k}} \geq {\bf v_{\pi_k}}$$
$${\bf v_{\pi_{k+1}}} = \lim_{N\rightarrow \infty} {\bf B}_{\pi_{k+1}}^N {\bf v_{\pi_k}} \geq {\bf v_{\pi_k}}$$
\end{itemize}
\end{frame}

\begin{frame}
\frametitle{Policy Iteration}
\begin{itemize}
\item We have shown that in iteration $k+1$ of Policy Iteration, ${\bf v_{\pi_{k+1}}} \geq {\bf v_{\pi_k}}$
\item If ${\bf v_{\pi_{k+1}}} = {\bf v_{\pi_k}}$, the above inequalities would hold as equalities
\item So this would mean $\bbs {\bf v_{\pi_k}} = {\bf v_{\pi_k}}$
\item But $\bbs$ has a unique fixed point $\vstar$
\item So this would mean ${\bf v_{\pi_k}} = \vstar$
\item Thus, at each iteration, Policy Iteration either strictly improves the VF or achieves the optimal VF $\vstar$
\end{itemize}
\end{frame}

\section{Value Iteration}

\begin{frame}
\frametitle{Value Iteration}
\begin{itemize}
\item $\bbs$ satisfies the conditions of Contraction Mapping Theorem
\item $\bbs$ has a unique fixed point $\vstar$, meaning $\bbs \vstar = \vstar$
\item This is a succinct representation of Bellman Optimality Equation
\item Starting with any VF $\bv$ and repeatedly applying $\bbs$, we will reach $\vstar$
$$\lim_{N\rightarrow \infty} \bbs^N \bv = \vstar \mbox{ for any VF } \bv$$
\item This is a succinct representation of the Value Iteration Algorithm
\end{itemize}
\end{frame}

\section{Policy Optimality}

\begin{frame}
\frametitle{Greedy Policy from Optimal VF is an Optimal Policy}
\begin{itemize}
\item Earlier we argued that ${\bf B}_{G(\bv)} \bv= \bbs \bv$ for any VF $\bv$. Therefore,
$${\bf B}_{G(\vstar)} \vstar = \bbs \vstar$$
\item But $\vstar$ is the fixed point of $\bbs$, meaning $\bbs \vstar = \vstar$. Therefore,
$${\bf B}_{G(\vstar)} \vstar = \vstar$$
\item But we know that ${\bf B}_{G(\vstar)}$ has a unique fixed point ${\bf v}_{G(\vstar)}$. Therefore,
$$\vstar =  {\bf v}_{G(\vstar)}$$
\item This says that simply following the deterministic greedy policy $G(\vstar)$ (created from the Optimal VF $\vstar$) in fact achieves the Optimal VF $\vstar$
\item In other words, $G(\vstar)$ is an Optimal (Deterministic) Policy
\end{itemize}
\end{frame}

\end{document}
