%%%%%%%%%%%%%%%%%%%%%%%%%%%%%%%%%%%%%%%%%
% Beamer Presentation
% LaTeX Template
% Version 1.0 (10/11/12)
%
% This template has been downloaded from:
% http://www.LaTeXTemplates.com
%
% License:
% CC BY-NC-SA 3.0 (http://creativecommons.org/licenses/by-nc-sa/3.0/)
%
%%%%%%%%%%%%%%%%%%%%%%%%%%%%%%%%%%%%%%%%%

%----------------------------------------------------------------------------------------
%	PACKAGES AND THEMES
%----------------------------------------------------------------------------------------

\documentclass[handout]{beamer}

\mode<presentation> {

% The Beamer class comes with a number of default slide themes
% which change the colors and layouts of slides. Below this is a list
% of all the themes, uncomment each in turn to see what they look like.

%\usetheme{default}
%\usetheme{AnnArbor}
%\usetheme{Antibes}
%\usetheme{Bergen}
%\usetheme{Berkeley}
%\usetheme{Berlin}
%\usetheme{Boadilla}
%\usetheme{CambridgeUS}
%\usetheme{Copenhagen}
%\usetheme{Darmstadt}
%\usetheme{Dresden}
%\usetheme{Frankfurt}
%\usetheme{Goettingen}
%\usetheme{Hannover}
%\usetheme{Ilmenau}
%\usetheme{JuanLesPins}
%\usetheme{Luebeck}
\usetheme{Madrid}
%\usetheme{Malmoe}
%\usetheme{Marburg}
%\usetheme{Montpellier}
%\usetheme{PaloAlto}
%\usetheme{Pittsburgh}
%\usetheme{Rochester}
%\usetheme{Singapore}
%\usetheme{Szeged}
%\usetheme{Warsaw}

% As well as themes, the Beamer class has a number of color themes
% for any slide theme. Uncomment each of these in turn to see how it
% changes the colors of your current slide theme.

%\usecolortheme{albatross}
%\usecolortheme{beaver}
%\usecolortheme{beetle}
%\usecolortheme{crane}
%\usecolortheme{dolphin}
%\usecolortheme{dove}
%\usecolortheme{fly}
%\usecolortheme{lily}
%\usecolortheme{orchid}
%\usecolortheme{rose}
%\usecolortheme{seagull}
%\usecolortheme{seahorse}
%\usecolortheme{whale}
%\usecolortheme{wolverine}

%\setbeamertemplate{footline} % To remove the footer line in all slides uncomment this line
%\setbeamertemplate{footline}[page number] % To replace the footer line in all slides with a simple slide count uncomment this line

%\setbeamertemplate{navigation symbols}{} % To remove the navigation symbols from the bottom of all slides uncomment this line
}

\usepackage{graphicx} % Allows including images
\usepackage{booktabs} % Allows the use of \toprule, \midrule and \bottomrule in tables
\usepackage{cool}
\usepackage{tikz}
\usepackage{amsmath}
\usepackage{pseudocode}
\usepackage{MnSymbol,wasysym}
\DeclareMathOperator*{\argmax}{argmax}
\DeclareMathOperator*{\argmin}{argmin}
\usetikzlibrary{positioning}

%----------------------------------------------------------------------------------------
%	TITLE PAGE
%----------------------------------------------------------------------------------------

\title[Adaptive Multistage Sampling]{Adaptive Multistage Sampling Algorithm: \\ The Origins of Monte Carlo Tree Search} % The short title appears at the bottom of every slide, the full title is only on the title page

\author{Ashwin Rao} % Your name
\institute[Stanford] % Your institution as it will appear on the bottom of every slide, may be shorthand to save space
{
ICME, Stanford University
 % Your institution for the title page
}

\date{\today} % Date, can be changed to a custom date

\begin{document}
\begin{frame}
\titlepage % Print the title page as the first slide
\end{frame}

\begin{frame}
\frametitle{Monte Carlo Tree Search (MCTS)}
\pause
\begin{itemize}[<+->]
\item MCTS was popularized a few years ago by \href{https://www.nature.com/articles/nature16961}{\underline{\textcolor{blue}{Deep Mind's AlphaGo}}}
\item It is a simulation-based method to identify the best action in a state
\item MCTS term was first introduced by \href{https://hal.inria.fr/inria-00116992/document}{\underline{\textcolor{blue}{Remi Coulom}}} for game trees
\item Each round of MCTS consists of four steps:
\begin{itemize}
\item Selection: Successively select children from root R to leaf L
\item Expansion: Create node C as a new child of L
\item Simulation: Complete a random playout from C
\item Backpropagation: Use result of playout to update nodes from C to R
\end{itemize}
\end{itemize}
\begin{figure}
\includegraphics[scale=0.115]{MCTS.png}
\end{figure}
\end{frame}

\begin{frame}
\frametitle{The Selection Step in MCTS}
\begin{itemize}
\item Selection involves picking a child with ``most promise''
\item This means prioritizing children with higher success estimates
\item For estimate confidence, we need sufficient playouts under {\em each} child
\item This is our usual {\em Explore v/s Exploit} dilemma (Multi-armed Bandit)
\item Explore v/s Exploit formula for games first due to \href{http://ggp.stanford.edu/readings/uct.pdf}{\underline{\textcolor{blue}{Kocsis-Szepesvari}}}
\item Formula called {\em Upper Confidence Bound 1 for Trees} (abbrev. UCT)
\item Most current MCTS Algorithms are based on some variant of UCT
\item UCT is based on UCB1 formula of \href{https://homes.di.unimi.it/cesa-bianchi/Pubblicazioni/ml-02.pdf}{\underline{\textcolor{blue}{Auer, Cesa-Bianchi, Fischer}}}
\item However, MCTS and UCT concepts first appeared in the \href{https://pdfs.semanticscholar.org/a378/b2895a3e3f6a19cdff1a0ad404b301b5545f.pdf}{\underline{\textcolor{blue}{Adaptive Multistage Sampling algorithm of Chang, Fu, Hu, Marcus}}}
\item Adaptive Multistage Sampling (AMS) is a generic simulation-based algorithm to solve a finite-horizon Markov Decision Process (MDP)
\item AMS can be considered as the ``spiritual origin'' of MCTS/UCT
\item Hence, this lecture is dedicated to AMS
\end{itemize}
\end{frame}

\begin{frame}
\frametitle{The Setting for the AMS Algorithm}
\begin{itemize}
\item MDP with finite number of time steps $t=0, 1, \ldots, T$
\item State denoted $s_t \in \mathcal{S}$, where $\mathcal{S}$ is very large
\item Action denoted $a_t \in \mathcal{A}$, where $\mathcal{A}$ is fairly small
\item Reward $r_t \in \mathbb{R}$, with $\mathbb{E}[r_t|(s_t, a_t)]$ provided as a function $R(s_t,a_t)$
\item Next time step's state $s_{t+1}$ can be generated by invoking a random sampling function $SF(s_t,a_t)$, i.e., $s_{t+1} = SF(s_t, a_t)()$
\item Discount factor denoted as $\gamma$, and $r_T = 0$
\item The problem is to calculate the Optimal Value function $V_t^*(s_t)$
\item Unlike tabular backward induction where state transition probabilities are given, here only a sampling function (for next state) is given
\item Armed with the sampling function, can we do better than backward induction for the case where $\mathcal{S}$ is very large and $\mathcal{A}$ is small?
\end{itemize}
\end{frame}

\begin{frame}
\frametitle{Outline of AMS Algorithm}
\begin{itemize}
\item AMS Algorithm is based on a fixed allocation of action selections for each state in each time step
\item Denote number of action selections per state in time step $t$ as $N_t$
\item Denote $\hat{V}_t^{N_t}(s_t)$ as the AMS Algorithm estimate of $V_t^*(s_t)$
\item Let $N_t^{s_t,a_t}$ be the number of selections of $a_t$ for $s_t$ ($\sum_{a_t \in \mathcal{A}} N_t^{s_t,a_t} = N_t$)
\item Proportions of $N_t^{s_t,a_t}$ based on {Explore v/s Exploit} UCT formula
\item For each of the $N_t^{s_t,a_t}$ selections of $a_t$, {\em one} next-state $s_{t+1}$ is sampled
\item Each $s_{t+1} = SF(s_t,a_t)()$ sample leads to recursive call $\hat{V}_{t+1}^{N_{t+1}}(s_{t+1})$
\item Optimal Action Value Function $Q_t^*(s_t, a_t)$ estimated as:
$$\hat{Q}_t^{N_t}(s_t,a_t) = R(s_t,a_t) + \gamma \cdot \frac {\sum_{j=1}^{N_t^{s_t,a_t}} \hat{V}_{t+1}^{N_{t+1}}(SF(s_t,a_t)())} {N_t^{s_t,a_t}}$$
\item $V_t^*(s_t) = \max_{a_t} Q_t^*(s_t,a_t)$ approximated as:
$$\hat{V}_t^{N_t}(s_t) = \sum_{a_t} \frac {N_t^{s_t,a_t}} {N_t} \cdot \hat{Q}_t^{N_t}(s_t,a_t)$$
\end{itemize}
\end{frame}


\begin{frame}
\frametitle{The AMS Algorithm}
\pause
\begin{pseudocode}{OptVF}{t,s,N_t}
\IF{t == T} \mbox{ } \RETURN 0\\
\COMMENT{Initialize VALS and CNTS by selecting each action once}\\
\FOR a \GETS \mathcal{A} \DO
\BEGIN
VALS[a] \GETS OptVF(t+1, SF(s,a)(), N_{t+1})\\
CNTS[a] \GETS 1
\END
\\
\FOR i \GETS  |\mathcal{A}| \TO N_t - 1 \DO
\BEGIN
\COMMENT{Pick action based on UCB1 {\em Explore v/s Exploit} formula}\\
a^* \GETS \argmax_{a \in \mathcal{A}} (R(s,a) + \gamma \cdot \frac {VALS[a]} {CNTS[a]} + \sqrt{\frac {2 \ln {i}} {CNTS[a]}}) \\
VALS[a^*] \GETS VALS[a^*] + OptVF(t+1, SF(s,a^*)(),N_{t+1}) \\
CNTS[a^*] \GETS CNTS[a^*] + 1
\END
\\
\RETURN {\sum_{a \in \mathcal{A}} \frac {CNTS[a]} {N_t} \cdot (R(s,a) + \gamma \cdot \frac {VALS[a]} {CNTS[a]})}\\
\COMMENT{$N_t$ next-state samplings and $N_t$ recursive calls to $OptVF$}\\
\end{pseudocode}
\end{frame}

\begin{frame}
\frametitle{Running Time, Bias, Convergence and Code}
\begin{itemize}
\item Let $N = \max{(N_0, N_1, \ldots, N_{T-1})}$ and assume $N > |\mathcal{A}|$
\item Running time of AMS Algorithm is of the order of $N^T \cdot |\mathcal{A}|$
\item Compare this versus backward induction running time of $|\mathcal{S}|^2 \cdot |\mathcal{A}| \cdot T$
\item So AMS is more efficient when $\mathcal{S}$ is very large (typical in real-world)
\item \href{https://pdfs.semanticscholar.org/a378/b2895a3e3f6a19cdff1a0ad404b301b5545f.pdf}{\underline{\textcolor{blue}{AMS paper}}} proves the estimate $\hat{V}_0^{N_0}(s_0)$ is asymptotically unbiased
$$\lim_{N_0\rightarrow \infty} \lim_{N_1\rightarrow \infty} \ldots \lim_{N_{T-1}\rightarrow \infty} \mathbb{E}[\hat{V}_0^{N_0}(s_0)] = V_0^*(s_0) \mbox{ for all } s_0 \in \mathcal{S}$$
\item AMS paper also proves that the worst-possible bias is bounded by a quantity that converges to zero at rate $O(\sum_{t=0}^{T-1} \frac {\ln N_t} {N_t})$
$$0 \leq V_0^*(s_0) - \mathbb{E}[\hat{V}_0^{N_0}(s_0)] \leq O(\sum_{t=0}^{T-1} \frac {\ln N_t} {N_t}) \mbox{ for all } s_0 \in \mathcal{S}$$
\item Here's some \href{https://github.com/coverdrive/MDP-DP-RL/blob/master/src/algorithms/ams.py}{\underline{\textcolor{blue}{Python code for the AMS Algorithm}}} you can play with
\end{itemize}
\end{frame}


\end{document}