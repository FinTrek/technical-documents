\documentclass[12pt]{exam}
\usepackage[utf8]{inputenc}
\usepackage{graphicx} % Allows including images
\usepackage{cool}
\usepackage{tikz}
\usepackage{amsmath}
\usepackage{listings}
\usepackage[colorlinks = true,
            linkcolor = blue,
            urlcolor  = blue,
            citecolor = blue,
            anchorcolor = blue]{hyperref}
\usepackage{pseudocode}
\usepackage{MnSymbol,wasysym}
\usepackage{geometry} % see geometry.pdf on how to lay out the page. There's lots.
\DeclareMathOperator*{\argmax}{argmax}
\DeclareMathOperator*{\argmin}{argmin}
\geometry{a4paper} 
\newgeometry{vmargin={20mm}, hmargin={14mm,18mm}}
 
\begin{document}
\begin{center}
{\large {\bf Stanford CME 241 (Winter 2020) - Final Exam Solutions}}
 
\end{center}
\vspace{5mm}
 
\begin{questions}

\question Consider an MDP with infinite number of states $\mathcal{S} = \{1,2,3, \ldots \}$ and finite number of actions $\mathcal{A} = \{A_1, A_2, \ldots, A_m\}$ for some fixed $m\geq 1$. The transition probabilities are as follows:
For all states $s \in \mathcal{S}$,
$$
Pr[(s+j)|(s,A_i)] =
\begin{cases}
\frac 1 i & \text{ for each } 1\leq j \leq i\\
0 & \text{ otherwise}
\end{cases}
$$
For all states $s, s' \in \mathcal{S}$ and for all actions $a \in \mathcal{A}$, $R(s,a,s')$ denotes the reward when transitioning from $s$ to $s'$ upon taking action $a$
$$R(s,a,s') = (s'-s)^n \text{ for some fixed } 1 \leq n \in \mathbb{R}$$
Let the discount factor be denoted by $0 \leq \gamma < 1$.
\begin{itemize}
\item {\bf 6 points}: Derive analytical expressions for the Optimal Value Function $V_*(\cdot)$ and for an Optimal Deterministic Policy (each in terms of $m, n, \gamma$).
\item {\bf 4 points}: Now assume that an agent, instead of following an optimal policy, decides to follow the stochastic policy: $\pi(A_i|s) = \frac 1 m$ for all $1 \leq i \leq m$ for all states $s \in \mathcal{S}$. Derive an analytical expression for the Value Function $V_{\pi}(\cdot)$ for this policy $\pi$ (in terms of $m, n, \gamma$).
\end{itemize}

{\bf Answer: } Let us write out the Bellman Optimality Equation.
$$V_*(s) = \max_{1\leq i \leq m} \{ \sum_{j=1}^i \frac {j^n + \gamma \cdot V_*(s+j)} {i}  \}$$ 
We note that since this has an infinite number of states, since the movements are only in one direction, and since transitions and rewards do not depend on the state $s$ (they depend on the action $A_i$ and on $s'-s$ respectively), $V_*(s)$ should be independent of $s$. So we write $V_*(s)$ as simply the single number $V_*$. So we rewrite the Bellman Optimality Equation as:
$$V_* = \max_{1\leq i \leq m} \{ \sum_{j=1}^i \frac {j^n + \gamma \cdot V_*} i  \}$$
$$\Rightarrow V_* = \max_{1\leq i \leq m} \{ \frac 1 i \sum_{j=1}^i j^n + \gamma \cdot V_*  \}$$
$$\Rightarrow V_* - \gamma V_* = \max_{1\leq i \leq m} \{ \frac 1 i \sum_{j=1}^i j^n \}$$

Since $j^n$ is monotonically increasing as a function of $j$, we note that:
$$\argmax_{1\leq i \leq m}  \{\frac1 i  \sum_{j=1}^i j^n \} = m$$.
Therefore,
$$(1-\gamma)V^* = \frac 1 m \sum_{j=1}^m j^n$$
$$\Rightarrow V_* = \frac 1 {m(1-\gamma)} \sum_{j=1}^m j^n$$
Therefore, the optimal policy $\pi_*: \mathcal{S} \rightarrow \mathcal{A}$ is given by:
$$\pi_*(s) = m \mbox{ for all } s \in \mathbb{Z}^+$$

Now we derive the analytical expression for the Value Function $V_{\pi}(\cdot)$ for the stochastic policy $\pi(A_i|s) = \frac 1 m$ for all $1 \leq i \leq m$ for all states $s \in \mathcal{S}$. We start by noting (based on the same argument as $V_*(\cdot)$) that $V_{\pi}(s)$ is independent of $s$. So we write $V_{\pi}(s)$ as simply the single number $V_{\pi}$. The Bellman Expectation Equation is:
$$V_{\pi} =  \sum_{i=1}^m \{ \frac 1 m \cdot \{ \sum_{j=1}^i \frac {j^n + \gamma \cdot V_{\pi}} {i}  \} \}$$
$$\Rightarrow V_{\pi} = \frac 1 {m(1-\gamma)} \sum_{i=1}^m \{ \frac 1 i \sum_{j=1}^i j^n \}$$

\vspace{5mm}

\question Assume you have fixed data available as $N$ complete episodes where each episode is in the form:
$$S_1,R_1,S_2,R_2,\ldots, S_n,R_n, T$$
where $S_1,S_2,\ldots,S_n$ is the sequence of states visited in the episode and $R_1,R_2,\ldots,R_n$ are the associated rewards following the visited states. $T$ is the terminal state. Note that $n$ (the length of an episode) can vary across the $N$ episodes. Note that there are no actions here, so the setting for this data is an underlying MRP and not MDP. Assume discount factor $\gamma = 1$.

{\bf Given only this data of fixed $N$ episodes, your task is to implement working Python code that will estimate the Value Function for discount factor $\gamma = 1$ based on a variety of methods}. An outline of the code is made available for you \href{https://github.com/coverdrive/MDP-DP-RL/blob/master/src/examples/exam_problems/mrp_tdmc_outline.py}{here}. A couple of functions (\lstinline{get_state_return_samples}, required for tabular Monte-Carlo, and \lstinline{get_state_reward_next_state_samples}, required for the other methods) have been implemented for you and you may use them as helper functions. Your task is to implement the following methods, each compatible with their respective function interfaces provided in this outline code. 

\begin{itemize}
\item {\bf Tabular Monte-Carlo - 3 points: } Implement \lstinline{get_mc_value_function}.
\item {\bf MRP - 6 points: } Implement \lstinline{get_probability_and_reward_functions} and using its output, implement \lstinline{get_mrp_value_function} (based on MRP Bellman Equation).
\item {\bf Tabular TD(0) with Experience Replay - 5 points: } Implement \lstinline{get_td_value_function}.
\item {\bf LSTD - 6 points: } Implement \lstinline{get_lstd_value_function}.
\end{itemize}

Make sure to read the comments/hints provided in the outline code (within each of the above functions you need to implement).

If you run the \lstinline{__main__} code, it will evaluate each of your 4 methods on the data that is set up in the \lstinline{__main__} code. In your answers-document, include the Python code you implemented and write the Value Function outputs you obtained for each of the 4 methods (up to 2 decimal places). {\bf Make sure your output Value Functions correspond to discount factor $\gamma = 1$}.

{\bf Answer: } The completed code is \href{https://github.com/coverdrive/MDP-DP-RL/blob/master/src/examples/exam_problems/mrp_tdmc.py}{here}.
The output upon running \lstinline{__main__} is as follows:
\begin{itemize}
\item Monte-Carlo - A: 9.57, B: 5.64
\item MRP - A: 12.93, B: 9,60
\item TD - A: 12.88, B: 9.56
\item LSTD - A: 12.93, B: 9.60
\end{itemize}

\vspace{3mm}

\question Assume you are the owner of a bank where customers come in randomly everyday to make cash deposits and to withdraw cash from their accounts. At the end of each day, you can borrow (from another bank, without transaction costs) any cash amount $y > 0$ at a constant daily interest rate $R$, meaning you will need to pay back a cash amount of $y(1+R)$ at the end of the next day. Also, at the end of each day, you can invest a portion of your bank's cash in a risky (high return, high risk) asset. Assume you can change the amount of your investment in the risky asset each day, with no transaction costs (this is your mechanism to turn any amount of cash into risky investment or vice-versa). The key point here is that once you make a decision to invest a portion of your cash in the risky asset at the end of a day, you will not have access to this invested amount as cash that otherwise could have been made available to customers who come in the next day for withdrawals. More importantly, if the cash amount $c$ in your bank at the start of a day is less than $C$, the banking regulator will make you pay a penalty of $K \cdot \cot(\frac {\pi \cdot c}{2C})$ (for a given constant $C > 0$ and a given constant $K > 0$).

For convenience, we make the following assumptions:
\begin{itemize}
\item  Assume that the borrowing and investing is constrained so that we end the day (after borrowing and investing) with positive cash ($c >  0$) and that any amount of regulator penalty can be immediately paid (meaning $c \geq K \cdot \cot(\frac {\pi \cdot c}{2C})$ when $c \leq C$).
\item Assume that the deposit rate customers earn is so small that it can be ignored.
\item Assume for convenience that the first half of the day is reserved for only depositing money and the second half of the day is reserved for only withdrawal requests.
\item Assume that if you do not have sufficient cash to fulfill a customer withdrawal request, you ask the customer to make the withdrawal request again the next day.
\item Assume all quantities are continuous variables.
\end{itemize}

\begin{itemize}
\item {\bf 5 points:} Your first task is to model an MDP so you can run the bank in the most optimal manner, i.e., maximizing the Expected Utility of assets less liabilities at the end of a $T$-day horizon, conditional on any current situation of assets and liabilities. Specify the states, actions, transitions, rewards with precise mathematical notation (make sure you do the financial accounting from one day to the next precisely).
\item {\bf 5 points:} In a practical setting, we do not know the exact probability distributions of the customer deposits and withdrawals. Neither do we know the exact stochastic process of the risky asset. But assume we have access to a large set of historical data detailing daily customer deposits and withdrawal requests, as well as daily historical market valuations of the risky asset. Assume we also have data on new customers as well as leaving customers (sometimes due to their withdrawal requests not being satisfied promptly). Describe your approach to solve this problem with Reinforcement Learning by using the historical data described above. Specify which Reinforcement Learning algorithm you would use, including any customizations for the specifics of this problem. Although you are not expected to write any code for this problem, provide sufficient detail that will enable a programmer with knowledge of RL to implement your ideas.
\end{itemize}

{\bf Answer: } Since there are no transactions costs with buying/selling shares or with borrowing cash, we assume that at the end of each day (after the bank has closed its doors for the day), the following sequence of events occur:
\begin{itemize}
\item Event E1: We sell the entire quantity of stock we purchased the previous day
\item Event E2: Then we make a decision to increase or decrease the quantity of cash-borrowing (i.e. decision to increase or decrease our liability)
\item Event E3: Then we make a decision to purchase a certain quantity of stock (to hold over the course of the next day)
\end{itemize}
In our MDP, we observe the {\em State} after event E1 and before event E2. The {\em State} will consist of 5 components:
\begin{itemize}
\item The date, denoted as $t \in \{1, 2, \ldots, T\}$
\item The net cash in the bank after selling the previous day's stock purchase, denoted as $x_t \in \mathbb{R}_{\geq 0}$
\item The liability (money currently owed to the bank from whom we borrow), denoted as $l_t \in \mathbb{R}_{\geq 0}$
\item The quantity of withdrawal requests we were not able to fulfill earlier in the day (backlogged withdrawals), denoted as $b_t \in \mathbb{R}_{\geq 0}$
\item The value of a single share of the stock, denoted as $s_t \in \mathbb{R}_{\geq 0}$
\end{itemize}

A key element in this problem is the fact that the regulator requires us to have a minimum amount of cash $c_t$ at the start of each day. Since $\cot$ is a monotonically decreasing function in $(0, \frac {\pi} 2]$), we have the condition $c_t \geq c_{min}$ where $c_{min} \in (0, C)$ is defined as:
$$c_{min} = K\cdot \cot(\frac {\pi \cdot c_{min}} {2C})$$

Now let us consider event E2. We denote $y_t \in \mathbb{R}$ as the further increase or decrease in our liability, meaning $y_t$ can be positive (meaning: borrow) or negative (meaning: pay off some of the current liability). Note that if we choose to pay off some of the current liability, there are some constraints. Firstly, we cannot make our net liability negative (we are not allowed to net-lend money to the bank from whom we borrow), hence $l_t + y_t \geq 0 \Rightarrow y_t \geq -l_t$. Secondly, we cannot let our net cash go below $c_{min}$, hence $x_t + y_t \geq c_{min} \Rightarrow y_t \geq c_{min} - x_t$. Combining these two inequalities, we get the following constraint on $y_t$:
$$y_t \geq \max(-l_t, c_{min} - x_t)$$

Now let us consider event E3. We denote $z_t\in \mathbb{R}$ as the number of shares of stock to purchase. But note that this is constrained by the fact that the net cash cannot fall below $c_{min}$. Therefore, $x_t + y_t - z_ts_t \geq c_{min}$ which sets a constraint on $z_t$ as:
$$0 \leq z_t \leq \frac {x_t + y_t - c_{min}} {s_t}$$
 
In our MDP, the {\em Action} is defined as the pair $(y_t,z_t)$ with the {\em Action Space} for $t < T$ defined by the constraints specified above for $y_t$ and $z_t$ (note: $y_T = z_T = 0$).

To specify the state transitions, we need to define a few random variables that will define the dynamics of the MDP.
\begin{itemize}
\item Deposits on day $t$ denoted by the random variable $d_t \in \mathbb{R}_{\geq 0}$
\item Withdrawal requests on day $t$ denoted by the random variable $w_t = f(t, b_{t-1})$ for $2 \leq t \leq T$ where $f: \{1, \ldots, T\} \times \mathbb{R}_{\geq 0} \rightarrow \mathbb{R}_{\geq 0}$ describes the usual (random) withdrawal requests as well as some of the previous day's unfulfilled withdrawal requests spilling to the current day
\item Value of a single share of stock on day $t$ denoted by the random variable $s_t = g(t, s_{t-1})$ for $2\leq t \leq T$ where $g: \{1, \ldots, T\} \times \mathbb{R}_{\geq 0} \rightarrow \mathbb{R}_{\geq 0}$ describes the stochastic movement of the stock price from one day to the next
\end{itemize}

Armed with these random variables, we are now ready to specify the state transitions.

$$x_{t+1} = \max(x_t + y_t - z_ts_t - K \cdot \cot(\frac {\pi \cdot \min(x_t + y_t - z_ts_t, C)} {2C})   + d_{t+1} - f(t+1, b_t), 0) + y_t \cdot g(t+1, s_t)$$
$$l_{t+1} = (l_t + y_t)(1+R)$$
$$b_{t+1} = \max(-(x_t + y_t - z_ts_t - K \cdot \cot(\frac {\pi \cdot \min(x_t + y_t - z_ts_t, C)} {2C})   + d_{t+1} - f(t+1,b_t)), 0)$$
$$s_{t+1} = g(t+1, s_t)$$


The {\em Reward} on day $t$ for $1\leq t \leq T-1$ is 0, and the {\em Reward} on day $T$ is $U(x_T - l_T)$ (where $U(\cdot)$ is the Utility function defined by the bank owner's risk-aversion). 


We'd like to solve this problem with Reinforcement Learning. Although there are many different ways by which we could approach solving this problem and many different potential RL algorithms, here are some ideas.
\begin{itemize}
\item We use the historical data on a) Deposits b) Withdrawals c) Daily Stock Price movements to build a model of the above MDP (i.e., state transition probabilities based on the statistically-estimated probabilities of deposits $d_t$, withdrawals $f(t,b_{t-1})$ and stock moves $g(t,s_{t-1})$). One can get more sophisticated here by recording various explanatory factors for deposits, withdrawals and stock moves (eg: Day of week, demographics of current customers, econometric factors etc.). We would then have to predict the future movements of some of these explanatory factors and use those to predict the probabilities of future deposits, withdrawals and stock moves. We would use this model of the MDP to generate a large set of simulation episodes that incorporate the above-mentioned estimated probabilities. 
\item Since the action space is large (large number of possible $(y_t,z_t)$ tuples), it would be suitable to employ a Policy Gradient algorithm. Secondly, the state space is also large, so will need a good function approximation for the $Q$-value function (that Policy Gradient requires). This would be an Actor-Critic algorithm. One would need to make a prudent choice of features for both the Actor and the Critic neural networks (although an initial prototype could be quickly built with linear-function approximations).
\end{itemize}
 
\end{questions}

\end{document}