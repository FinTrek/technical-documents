
\documentclass[12pt]{amsart}
\usepackage{geometry} % see geometry.pdf on how to lay out the page. There's lots.
\geometry{a4paper} % or letter or a5paper or ... etc
% \geometry{landscape} % rotated page geometry

% See the ``Article customise'' template for come common customisations

\title{Optimal Asset Allocation in Discrete Time}
\author{Stanford University - CME 241 Assignment Problem}
\date{} % delete this line to display the current date

%%% BEGIN DOCUMENT
\begin{document}

\maketitle

We are given wealth $W_0$ at time 0. At each of discrete time steps labeled $t = 0, 1, \ldots, T$, we are allowed to allocate the current wealth $W_t$ in a risky asset and a riskless asset in an unconstrained, frictionless manner. The risky asset yields a random rate of return $\sim N(\mu, \sigma^2)$ over each single time step. The riskless asset yields a rate of return denoted by $r$ over each single time step.

Our goal is to maximize the Utility of Wealth at the final time step $t=T$ by dynamically allocating $x_t$ in the risky asset and the remaining $W_t - x_t$ in the riskless asset for each $t = 0, 1, \ldots, T-1$ (assume no transaction costs and no restrictions on going long or short in either asset). Assume the single-time-step discount factor is $\rho$ and the Utility of Wealth at the final time step $t=T$ is $U(W_T) = - \frac {e^{-a W_T}} {a}$ for some fixed $a > 0$.

\begin{itemize}
\item Formulate this problem as a {\em Continuous States}, {\em Continuous Actions} MDP by specifying it's {\em State Transitions}, {\em Rewards} and {\em Discount Factor}. The problem then is to find the Optimal Policy.
\item As always, we strive to find the Optimal Value Function. The first step in determining the Optimal Value Function is to write the Bellman Optimality Equation.
\item Assume the functional form for the Optimal Value Function is $b_t e^{-c_t W_t}$ where $b_t, c_t$ are unknowns functions of only $t$.
\item Express the Bellman Optimality Equation using the above functional form for the Optimal Value Function.
\item Since the right-hand-side of the Bellman Optimality Equation involves a $\max$ over $x_t$, we can say that the partial derivative of the term inside the $\max$ with respect to $x_t$ is 0.
\item This enables us to write the Optimal Allocation $x_t^*$ in terms of $c_{t+1}$
\item Substituting this maximizing $x_t^*$ in the Bellman Optimality Equation enables us to express $b_t$ and $c_t$ as recursive equations in terms of $b_{t+1}$ and $c_{t+1}$ respectively.
\item We know $b_T$ and $c_T$ from the knowledge of the MDP {\em Reward} at $t=T$ (Utility of Terminal Wealth), which enables us to unroll the above recursions for $b_t$ and $c_t$.
\item Solving $b_t$ and $c_t$ yields the Optimal Policy and the Optimal Value Function.
\end{itemize}

\end{document}