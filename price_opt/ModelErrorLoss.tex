   
\documentclass[12pt]{amsart}
   \usepackage{geometry} % see geometry.pdf on how to lay out the page. There's lots.
   \geometry{a4paper} % or letter or a5paper or ... etc
   % \geometry{landscape} % rotated page geometry
   
   % See the ``Article customise'' template for come common customisations
   
   \title{Value Gap due to Model Error}
   \author{Ashwin Rao and Ajay Nerurkar}
   \date{} % delete this line to display the current date
   
   %%% BEGIN DOCUMENT
   \begin{document}
   
   \maketitle 
   \section{Introduction}
   This is a brief article to set the foundation for exploring the gap in Returns due to Model Error within the context of a Markov Decision Process. We consider a very simple example of Price Optimization to illustrate these concepts.
   
   \section{Notation and Abstract Setting}
   Let us say that we have estimated a model $M_1$. Solving (optimizing) the MDP with this model $M_1$ gives us the policy $\pi_1^*$ and the corresponding Value Function $V_1^*$. Now let us say that afterwards we come to know the true model is $M_2$. Applying policy $\pi_1^*$ on the true MDP (with true model $M_2$) gives us the Value Function $V_2$. Had we known the true model $M_2$ beforehand, we would have instead arrived at the optimal policy $\pi_2^*$ and the optimal Value Function $V_2^*$. We want to understand the gaps between the three value functions: $[V_1^*, V_2, V_2^*]$.
   
   \begin{itemize}
   \item We can conceptualize the gap between $V_1^*$ and $V_2^*$ as that due to ``Model Estimation Inadequacy''
   \item We can conceptualize the gap between $V_1^*$ and $V_2$ as that due to ``Outcome Surprise''
   \item We can conceptualize the gap between $V_2$ and $V_2^*$ as the slippage due to ``Lack of Prior Information''
   \end{itemize}
   
   We also want to understand the gap between the optimal policy $\pi_1^*$ and $\pi_2^*$.
       
    \section{Price Optimization}
    For ease of illustration of the above concepts, consider a very simple case of single time-step Price Optimization, an exponential model of Price Elasticity of Demand, and little chance of lost sales.

    
    Let the model $M_1$ be given by the demand function $f_1(p) = \alpha_1 \cdot e^{-\beta_1 p}$. The single time-step revenue for a choice of price $p$ is: $p \cdot f_1(p) = \alpha_1 \cdot p \cdot e^{-\beta_1 p}$. The optimal price (for maximum revenue) is given by:
   $$p_1^* = \frac 1 {\beta_1}$$ and the maximum revenue (optimal value function) is given by:
   $$V_1^* = \frac {\alpha_1} {\beta_1 \cdot e}$$
   
   Now let us say that the true model $M_2$ is given by the demand function $f_2(p) = \alpha_2 \cdot e^{-\beta_2 p}$. Executing the previously identified optimal policy for $M_1$ (i.e., $p_1^*$) on the true MDP (with true model $f_2(p)$) give us the value function:
   $$V_2 = p_1^* \cdot (\alpha_2 \cdot e^{- \beta_2 p_1^*}) = \frac {\alpha_2} {\beta_1} \cdot e^{-\frac {\beta_2} {\beta_1} }$$
   
   We also know that:
    $$p_2^* = \frac 1 {\beta_2}$$
    $$V_2^* = \frac {\alpha_2} {\beta_2 \cdot e}$$
    
    Now let us look at the percentage changes between $[V_1^*, V_2, V_2^*]$ and the percentage change between $p_1^*$ and $p_2^*$.

    \begin{itemize}
    \item $1 - \frac {V_2} {V_2^*} = 1 - \frac {\beta_2} {\beta_1} \cdot e^{1 - \frac {\beta_2} {\beta_1}}$
    \item $1 - \frac {V_2} {V_1^*} = 1 - \frac {\alpha_2} {\alpha_1} \cdot e^{1 - \frac {\beta_2} {\beta_1}}$
    \item $1 - \frac {V_2^*} {V_1^*} = 1 - \frac {\alpha_2 \beta_1} {\alpha_1 \beta_2}$
    \item $1 - \frac {p_2^*} {p_1^*} = 1 - \frac {\beta_1} {\beta_2}$
    \end{itemize} 
    
    Now let us define $\alpha_2$ as a small perturbation of $\alpha_1$ and $\beta_2$ as a small perturbation of $\beta_1$.
    Let $\frac {\alpha_2} {\alpha_1} = 1 + \delta_\alpha$ and let $\frac {\beta_2} {\beta_1} =  1 + \delta_\beta$. Then,
    
    \begin{itemize}
    \item $1 - \frac {V_2} {V_2^*} = 1 - (1 + \delta_\beta) \cdot e^{-\delta_\beta} \approx 1 - (1 + \delta_\beta) (1 - \delta_\beta + \frac {\delta_\beta^2} 2) =  \frac {\delta_\beta^2} 2$
    \item $1 - \frac {V_2} {V_1^*} = 1 - (1 + \delta_\alpha) \cdot e^{-\delta_\beta} \approx 1 - (1 + \delta_\alpha) (1 - \delta_\beta) =  \delta_\beta - \delta_\alpha$
    \item $1 - \frac {V_2^*} {V_1^*} = 1 - \frac {1 + \delta_\alpha} {1+\delta_\beta} \approx 1 - (1 + \delta_\alpha)(1 - \delta_\beta)  \approx \delta_\beta - \delta_\alpha$
    \item $1 - \frac {p_2^*} {p_1^*} = 1 - \frac 1 {1 + \delta_\beta} \approx 1 - (1 - \delta_\beta) =  \delta_\beta$
    \end{itemize}
    
   
   \end{document}